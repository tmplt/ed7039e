\documentclass[twocolumn]{article}
\usepackage[english]{babel}
\usepackage[T1]{fontenc}
\usepackage{graphicx}
\usepackage{kpfonts}[maths]
\usepackage{libertine}
\usepackage{placeins}
\usepackage{gensymb}
\usepackage{inconsolata}
\usepackage{subcaption}
\usepackage{mdframed}
\usepackage{caption}
\usepackage{amsmath}
\usepackage{lipsum}
\usepackage{tikz}
\usepackage{minted}
\usepackage{lipsum}
\usepackage{xcolor}
\usepackage{sectsty}
\usepackage{hyperref}
\usepackage{csquotes}
\usepackage[style=authoryear]{biblatex}
\usepackage[inline]{enumitem}


\newlist{inline-enum}{enumerate*}{1}
\setlist[inline-enum]{label=(\roman*)}

\hypersetup{
    colorlinks=true,
    urlcolor=blue,
    linkcolor=black,
    citecolor=black,
}

\definecolor{CERNblue}{HTML}{22529e}
\definecolor{CodeBg}{rgb}{0.95, 0.95, 0.95}

\setminted{
    % linenos=true,
    autogobble=true,
    frame=single,
    % framesep=2mm,
    % framerule=0.4pt,
    tabsize=4,
    fontsize=\footnotesize,
    breaklines,
    bgcolor=CodeBg,
}

\setlist[enumerate]{itemsep=0mm}
\setlist[itemize]{itemsep=-0mm}


% \sectionfont{\color{CERNblue}}
% \subsectionfont{\color{CERNblue}}
% \subsubsectionfont{\color{CERNblue}}

\newcommand{\hypermail}[1]{\href{mailto:#1}{\texttt{<#1>}}}

\addbibresource{ref.bib}

% TODO: add a link to the repository somewhere

\title{
D7039E/E7032E\footnote{Codes of courses taken by team members at Luleå Technical University}
--- Project: Model, Simulation and Implementation of a Mechatronical System for Use in a Mimicked Industrial Environment
}
\author{
As authored by:\footnote{In no particular order; All team members can be contacted via \href{mailto:vikson-6@student.ltu.se;lukkar-4@student.ltu.se;sansim-6@student.ltu.se;olemis-6@student.ltu.se;rubasp-6@student.ltu.se;alinou-6@student.ltu.se;alikhar-6@student.ltu.se}{this hyperlink}.} \\
Viktor Sonesten \hypermail{vikson-6@student.ltu.se} \\
Lukas Karlsson \hypermail{lukkar-4@student.ltu.se} \\
Simon Sandberg \hypermail{sansim-6@student.ltu.se} \\
Oleksiy Mishchenko \hypermail{olemis-6@student.ltu.se} \\
Ruben Asplund \hypermail{rubasp-6@student.ltu.se} \\
Ali Nouri \hypermail{alinou-6@student.ltu.se} \\
Ali Khademi \hypermail{alikha-6@student.ltu.se}
}
\date{\today}

\begin{document}
\maketitle

\section{Introduction}
\subsection{Background}
\label{intro-background}
% Introduce ARTEMIS and the Arrowhead project
The ARTEMIS\footnote{\textbf{A}dvanced \textbf{R}esearch \& \textbf{T}echnology for \textbf{EM}bedded \textbf{I}ntelligent \textbf{S}ystems} Industry Association is an organisation with the goal of innovating the embedded intelligent systems sector in Europe.~\parencite{artemis}
One of the approaches to reach this goal was the 2013 start of the ARROWHEAD project which ``[address] the efficiency and flexibility at the global scale by means of collaborative automation for five application verticals; [one of which being] production (manufacturing, process, energy).''~\parencite{arrowhead-project-call}

% Summarize the result of the project
The project concluded 2017 and yielded the Arrowhead Framework which enable interoperability between Internet of Things (IoT) devices by providing guarantees for:
\begin{inline-enum}
    \item real-time data handling;
    \item data and system security;
    \item automation system engineering; and
    \item scalability of automation systems.
\end{inline-enum}
\parencite{arrowhead-project-about}
With the above guarantees provided by Arrowhead, resources otherwise required for developing a new framework (ad-hoc or not) may instead be allotted to the system the framework ultimately will interact with.

% Describe the reason for this project.
% TODO: extend/rewrite; describe that the process of moving an object from a to b by use of robots have not been solved yet (mentioned during the second lecture).
The reason for this project is then to verify that Arrowhead can indeed be used to resolve a common industry problem: production; specifically the act of moving a object from one station to another,
which is a common procedure in product pipelines.

% TODO: mention that Arrowhead aims to create de-facto standards for European industrialization efforts?

% XXX: Currently disjointed from the background section: background has large focus on Arrowhead. This section does not.
\subsection{Problem description}
% Summarize the problem and describe that we're simulating a factory settings.
The problem this project aims to solve is that of automatically moving an object from a designated pick-up point to a designated drop-off point on a system-external signal.
In practise, the problem space aims to mimic the environment of a factory floor where an object under manufacture has been fully processed in a station and should be moved to the next.

% Boil the issue down to simple mathematical terms.
Described in different terms, the problem is the automatic displacement of an object from an origin point $(x_0, y_0, z_0)$ to a destination point $(x, y, z)$,
where $(x, y)$ denote a position on the factory floor and $z, z \geq 0$, an elevation above the floor.

% Mention the inclusion of Arrowhead in the problem.
A station signals a fully processed object via the Arrowhead Framework in a defined local cloud.
Upon this signal, the task of moving the object is to be delegated to the mechatronical system this project aims to develop.

% Draw a nice state machine representation of the issue.
From the above problem description five possible system states are thus argued for:
when the robot is
\begin{inline-enum}
\item waiting for a command;
\item moving to a destination;
\item following a navigation line;
\item picking an object up; and
\item dropping an object off.
\end{inline-enum}
The relation between these states, and the allowed transitions between them is visually described using the state machine in Fig.~\ref{fig:state_machine}.
\begin{figure*}[h]
  \centering
  \begin{tikzpicture}
    % TODO transpose?
    \node[state, initial, accepting] (wait) {waiting for \\ command};
    \node[state, right of=wait] (move) {moving to \\ destination};
    \node[state, below of=move] (follow) {following \\ line};
    \node[state, right of=follow] (pickup) {picking \\ up object};
    \node[state, left of=follow] (dropoff) {dropping \\ off object};

    \path[->] (wait) edge node{Arrowhead \\ signal} (move)
    (move) edge[sloped] node{line detected \\} (follow)
    (follow) edge[sloped] node{station detection \\ (system w/o object)} (pickup)
    (follow) edge[sloped] node{station detection \\ (system w/ object)} (dropoff)
    (pickup) edge[sloped] node{object picked up \\} (move)
    (dropoff) edge[sloped] node{object dropped \\} (wait)
    ;
  \end{tikzpicture}
  \caption{High-level state machine of the system.}
  \label{fig:state_machine}
\end{figure*}

% TODO: insert an image of the conveyor belt with measures.

\subsection{Limitations}
\label{intro-limits}
% Note and expound on the very large problem domain.
The problem domain is vast and infinite configurations can mimic a wanted factory setting:
\begin{inline-enum}
    \item different placement/rotations of one or more stations;
    \item obstacles of multiple types; static and dynamic (e.g. humans, other robots);
    \item recovery/discard of a dropped object; and
    \item conditional and/or time-dependent origin/destination points.
\end{inline-enum}
Support for more configurations require a more complex system model, simulation and implementation;
and more complexity requires more development time.

% Note time limitation and the smaller problem space we aim to tackle.
The system developed in this project is to be thoroughly modelled, simulated, and implemented in the time span of two semesters.
For that reason (and those stated in the previous paragraph), this project is thus limited to two stations:
a single designated pick-up point and a single designated drop-off point  --- both static and elevated (however, $z_0 = z$).
Furthermore, the system environment will contain no other obstacles, and if an object is dropped, the displacement procedure ends and is considered to have failed;
this project aims to implement a proof-of-concept solution for the stated problem, not a readily available product ready for industry integration.
If an object is successfully displaced automatically to its destination once, the project is considered a success.

% Note that the system has no real-time requirement
% XXX: unless we require the system to not overshoot its 2D destionations, then we need to process location data with known delay.
% TODO: find a reference on this statement.
Because the above defined problem domain lacks any time component (as would be the case if the system would need to dodge moving obstacles, respond to time-dependent destinations, etc.) and becafuse the object need only be moved to its destination eventually, the implementation is under no real-time requirement.
This greatly lessens the requirement of the system model and software implementation.

\subsection{Arrowhead framework}
The three mandatory core systems when running Arrowhead framework are; Service Registry, Orchestrator and Authorization.
The Gateway and Gatekeeper is needed when communicating with other clouds (inter-cloud).
All information in this section is from \url{https://github.com/eclipse-arrowhead/core-java-spring}.

\subsubsection{Service Registry}

The Service Registry holds the services and IP address and port to where the services is located. The information is stored in a SQL database.
Service Registry have three methods, register, unregister and query. The query method translates system name into information about physical endpoint; IP address and port.

\subsubsection{Authorization}
The Authorization system have two methods.
Authorization control are controlling which clients are allowed to consume a service. This is controlled through intra- and inter-cloud rules. The information is stored in a SQL database.
The other method is token generation which is used for accessing services within cloud.

\subsubsection{Orchestrator}
The Orchestrator have two methods.
When a application requsts a service, the orchestrator will look up if the application system is allowed to access that service.
If allowed it will send back the IP address and port.
The other methods is used when the IP address and port are changed for a service.
The Orchestrator will send the new information for the services which are connected.

There are two types of orchestration; dynamic orchestration and store orchestrator.
Dynamic orchestrator uses the application system name and authorization properites to find the providers.
Store orchestration uses the application systems's id to search the orchestration store database for providers.

\subsubsection{Gatekeeper}
The Gatekeeper have two methods; Global Service Discovery (GSD) and Inter-Cloud Negotiation (ICN).
GSD is looking via Relay systems for other clouds and looks for a provider which are serving a specific service.
After GSD the ICN process is used for establishing a way of collaboration between clouds.

\subsubsection{Gateway}
The Gateway system have two methods; connecting to a consumer and connection to a provider.
The two methods is used during the ICN process to establish a new datapath.
\section{System design and composition}
\subsection{Model}
The system is modelled in two parts: the moving base and the\ldots
% TODO: describe the moving base, and the attachment, whatever it ends up being.
% Include physical models.

\subsection{Simulation}
% Simulte the models from the previous section and show that it will work.
% Motivate regulation approach.

\subsection{Hardware}
% Explain the raspberry pi and it's attachments.

\subsection{Software}
\subsubsection{Reproducible system image generation}
% Talk about Nix here and all the repository's *.nix files.

\section{Implementation}
The git repository used throughout this project is publicly available at \href{https://github.com/tmplt/ed7039e}{Github/tmplt/ed7039e} \parencite{repo}.
If not otherwise specified, any references to a repository shall mean this repository.

% We chose to implement our System on a Raspberry Pi.
% This means our system is not in real-time (Linux too complex, other reasons)
% Allows people not versed in embedded systems to write implementations
% A proper implementation would be on a micro-controller that allows code to be run bare-metal, without having to fight with the Linux kernel.

\subsection{Milestones}
The project is divided into four milestones:
\begin{enumerate}
\item \textbf{Two-dimensional navigation:}
  the system should be able to determine its coordinates in a ad-hoc, local grid.
  From its initial position, it should then be able to respond to movement commands on the form ``move to position $(x, y)$''.

\item \textbf{Navigation-line detection:}
  using the subsystem for two-dimensional navigation, the system is to cross a line on the floor,
  thus detecting it and follow it towards the station.

\item \textbf{Station proximity detection, object pickup:}
  once the navigation-line is being followed, the system is to sense when it is sufficiently close to the station to readily use its arm to pick the object up.

\item \textbf{Object displacement, dropp-off:}
  after the object has been picked up, the system is to move to another station, find its navigation-line, follow it, and drop the object.
  Note that this milestone is a permutation of the combination of the previous milestones: the same phases should be done in the same order,
  but the system is to move to the second station instead and execute the pickup-process in reverse.
\end{enumerate}

% TODO: write here when each milestone was reached and why it took the time it did

\subsection{Prototyping}
% Here we describe the prototyping stages of the system's components if anything
% out of the ordinary pops up.

\subsubsection{NixOS}
% NixOS is unconventional
NixOS is an unconventional Linux distibution (henceforth following the common vernacular of ``distro'').

% Non-NixOS mutate global system state. Multi-version is thus tricky.

% Nix makes no assumption about global system state. No dependency hell. /nix/store/hash-name immutable.
% Nix lets you compose software ad build-time with maximum flexibility.

% Summarize what NixOS is and what it aims to provide
NixOS is a Linux-distribution with the ability to declare a certain system in a functional manner.
It is buit upon the Nix package manager that, in summary, aims to be
\begin{inline-enum}
\item reproducible:
  ``packages [are built] in isolation from each other. [\ldots] they are reproducible and don't have any undeclared dependencies.''\footnote{A positive side-effect of this feature is the complete mitigation of ``dependency hell''.};
  so if a package works on one machine, it will work on any machine.
  Also, when building a package on multiple machine, all machines will yield the exact same ouput file tree.
\item declarative:
  packages are described in expressions that are trivially shared and combined with other package declarations.
\item reliable:
  ``installing or upgrading one package cannot break other packages''.
\end{inline-enum}~\parencite{nixos.org}

% How does NixOS enable the above funtionality?
To enable the preceeding features, everything provided to and by Nix is stored on a Nix-enabled system under \texttt{/nix/store/<hash>-<pkg>},
where \texttt{<hash>} is a SHA256 checksum that provides a unique identifier to the package with name \texttt{<pkg>}.
If any dependency for the package (or if the build procedure) is in any way changed, a wholly new checksum is generated.
This Nix store is mounted as read-only to make its content immutable.
A usage of any package in the Nix store is manifested in the system's file system as a symbolic link.

% How does NixOS differ from conventional distros?
Because all files that are built and ultimately used are stored under \texttt{/nix/store}, NixOS breaks the filesystem hierarchy standard (FHS).
% TODO: what does this entail?

% Problems with hardware features on the Raspberry Pi. Simply not provided out of the box.
% A study of Raspbian would likely answer all questions in due time. But time consuming.

% Link to <https://github.com/NixOS/nixpkgs/pull/79370> for credit.
% Compare approach with Raspbian: mention that only a raspbian fork is officially supported for the BrickPi3.
On Raspbian, a simple \texttt{echo 'dtparam=spi=on' >> /boot/config.txt} and system reboot enables SPI and thus the ability to communicate with the BrickPi3.
On NixOS however, this file (nor any equivalent) exists, because of the wholly different design philosophies with the Debian-based Raspbian.
The ``Nix-appoach'' is instead...

\subsubsection{Decawave}
This project uses Decawave (or more specifically: a DWM1001 development board) for two-dimensional positioning.
The development board constitutes of the ultra wide-band module itself, the DWM1001C, an accelerometer,
and a Raspberry Pi-compatible GPIO-header.
Via communication over UART, we are able to query its measured position as reported by help of the Decawave anchor nodes\footnote{An anchor is another development board configured for static installation at a known coordinate, used for position calculation. A non-static device with unknown coordinates (as is used in our system) is known as a tag.},
and use this data to figure out where our system is located in the mimicked factory.

Over UART the device exposes a type-lengh-value (TLV) API which we query to recieve the tuple of
$(x, y, z, q)$, where
\begin{description}
\item[$(x, y, z)$] is the reported coordinate in millimeters in three-dimensional space, and
\item[$q$] is the quality factor: a measure of how sure the device is of the coordinates.
\end{description}

Of note is that the decive cannot approximate its position unless it can connect to at least three anchors.
Additionally, the quality factor, $q$, is higher when connected to four anchors.
During measurements, and if there are more than four anchors available, it will chose the best four anchors to calculate its position. % What does "best" mean here? Refer to documentation.
Because of this, we will then have to consider that our system may suddenly no longer be able fo find its position,
and decide what should be done in order to establish a connection with the disconnected anchors.

% The data can be considered a random process.

% (X, Y, Z, Q); how is Q calculated?
% What should we do if we cannot connect to 4 anchors at once, a wait?
% Mention that:
% - we have to account for the fact when we tag cannot connect to at least 3 anchors.
% - Qualitative data depends a lot on the positioning of the anchors
% - Built-in 3-axis accelerometer
% - Raspberry Pi compatible GPIO header. Communication via UART.
% - How should we interpset data? It is random proccess? Can we consider noise gaussian?

% TODO:
% RPi UART problems
% No access to acceleration data using TLV API; generic shell required.

\subsubsection{BrickPi3}
The BrickPi3 is a peripheral that allows a Raspberry Pi to work with LEGO Mindstorms hardware.
It works by communicating via the SPI function pins of the Raspberry Pi.
The recommended way to install all necessary components is via a \texttt{curl -k | bash}.
There are a few issues with this approach:
\begin{inline-enum}
\item \texttt{-k} is an alias for \texttt{--insecure};
  the recommended approach is thus to not verify the server certificate ---
  this allows a bad actor to feed you malicious code if they have access to your DNS or the target domain.
\item A \texttt{curl | bash} is bad practise for installation purposes as it commonly installs files that are disconnected from the system's package manager.
\item A \texttt{curl | bash} can be detected server-side and thus can conditionally feed a user malicious code.
  A dowload of the code first may thus pass a manual inspection before execution. \parencite{curl-bash}
\end{inline-enum}

Because we use NixOS, the content of the script had to be inspected so that an equivalent Nix expression could be written ---
see \texttt{nix/brickpi3.nix} for the final result.
Upon inspection, a few oddities stood out. The script:
\begin{enumerate}
\item expects and requires the script to be run by the user \texttt{pi}\footnote{Not all users of the peripheral is \texttt{pi}. For example, we use it as \texttt{root} while prototyping.};
\item changes the ownership of a directory with \texttt{sudo(8)} on files under \texttt{/home/pi}, to \texttt{pi}\footnote{In this context, the operations could all have been done as \texttt{pi}.};
\item insecurely downloads multiple scripts and executes them silently --- the downloaded scrips do the same;
\item configures an \texttt{apt(8)} repository (and thus requires to be run on a Debian derivative) for \texttt{npm(1)},
  the Node JavaScript package manager, but never installs or executes any JavaScript packages;
\item installs a C++ source file under \texttt{/usr/local/include}\footnote{A proper installation would be to build a shared library which can then be dynamically linked to when using the C++ drivers.};
\item downloads a precompiled version of \texttt{openocd(1)}, a on-chip debugger and programmer,
  and copies the files into system directories\footnote{No changes are made to the software, according to the mirror's documentation. An installation should instead then be made with the package manager, which is otherwise used in the scripts to install other components}, and then never uses it;
\item runs \texttt{git(1)} as a privileged user, sometimes.
\end{enumerate}
The above list is truncated for sake of brevity.

After a thorough manual inspection of all scripts it was found that only a single Python library (with a single dependency) had to be installed.
The final Nix expression is thus a combination of two \texttt{python3Packages.buildPythonPackage} where both sources are securely downloaded from official mirrors and verified with a known checksum.
We conclude that the usage of this Nix expression leaves the system in a proper state (which the official installation script does not, by oddity 5 and 6\footnote{we consider a proper state of system one in which all installed software components are tracked by the package manager(s).}) and greatly decreases the number of attack vectors with which to run malicious code on our system.

% What need to be done to be able to stack BrickPi3s?

\subsubsection{RobotOS}
% We wanted to use ROS as it was very common to the problem space, and had a lot of readily available solutions for common robot problems.
It was initially decided that the robot system would be implemented with RobotOS (ROS), ``a set of software libraries and tools that help you build robot applications.''\footnote{See \href{https://www.ros.org/}{https://www.ros.org/}.}
The chief reason was its common application in the problem space, its API for communicating different types of messages between different programs\footnote{Known as inter-process communication (IPC).} (in this context known as ``nodes''), and the many readily available solutions to problems we were likely to stumble upon.
% Only officially supports very specific Ubuntu versions, and while probably very applicable to use Nix in this case, it was deemed
% composing ROS on Nix would take too long. (the dependency tree is HUGE)
However, ROS is only officially supported on very specific versions of Ubuntu (at this time of writing), a disto we were not using and a disto that had a very different design philosophies from NixOS;
using ROS on NixOS would thus require a Nix expression to be written that correctly packages the software.
At this point, it was surmised that ROS made several assumption about the global system state that had to be adressed during packaging.
This reason alone would likely require a lot of prototyping time for a simple proof-of-concept execution.
A consultation from another effort to port ROS to an unofficial repository showed that a full desktop installation is constituted of up to 460 packages.\footnote{See \href{https://github.com/ros-noetic-arch}{https://github.com/ros-noetic-arch}, which packages RobotOS to Arch Linux, a distro that is not officially supported by the RobotOS project. Each repository corresponds to a ROS package.}
It was thus decided to find an alternative to RobotOS due to time constrains.

\subsubsection{LCM}
% Trivial to package: just a simple mkDerivation. Nodes are similarly easily packaged. See `nix/software-nodes.nix`.
Lightweight Communications and Marshalling (LCM) ``is a set of libraries and tools for message passing and data marshalling [\ldots] It provides a publish/subscribe message passing model and automatic marshalling/unmarshalling code generation with bindings for applications''.
LCM effectively provides a set of simple functions that enable IPC with the benefit of not requiring a special-purpose daemon (as is required when running ROS).
In difference to ROS, LCM supports any GNU/Linux system (and thus NixOS) and relies on UDP multicasting for broadcasting purposes.
Its short list of dependencies made LCM trivial to package to NixOS: the final expression in \texttt{nix/software-nodes.nix} can be summarized as a \texttt{stdenv.mkDerivation} and the whitelisting of an UDP port in the system firewall.

% Support for C and Python which we have decided to use thus far.
% The core component of ROS we wanted was the message-passing (IPC) component, which this library provides for ANY POSIX-compliant system.
Thus, as LCM:
\begin{inline-enum}
\item enables us to trivially utilize IPC with different message types;
\item has bindings for C and Python; and
\item is trivially packaged,
\end{inline-enum}
it was decided that our robot system would be implemented with help of it in place of ROS.

\section{Results}
\subsection{Performance of arm movement controllers}
Different controllers that was made for the motion control of the robot arm was compared when acting on a \(40^{\circ}\) angular error step response for vertical motion and \(60^{\circ}\) angular error step response for horizontal motion. In the following subsections graphs are presented where the evolution of the error is plotted against time for each controller. Rise time and settling time is shown for each test where rise time is the time measured at the error reaching \(5\%\) of the step value and settling time is measured at the error settling at \(2\%\) of the step value.

\subsubsection{P controller}
\begin{figure}[H]
\centering
\includegraphics[width = 13 cm]{Vertical_P_controller.png}
\caption{Evolution of vertical angular error from \(40^{\circ}\) step with P-controller}
\label{vert_P}
\end{figure}
Both rise time and settling time of the vertical P controller is 0.4 seconds.
\begin{figure}[H]
\centering
\includegraphics[width = 13 cm]{Horizontal_P_controller.png}
\caption{Evolution of horizontal angular error from \(60^{\circ}\) step with P-controller}
\label{vert_P}
\end{figure}
Rise time of horizontal P controller is 0.6 seconds and settling time is 1.6 seconds

\subsubsection{PI controller}
\begin{figure}[H]
\centering
\includegraphics[width = 13 cm]{Vertical_PI_controller.png}
\caption{Evolution of vertical angular error from \(40^{\circ}\) step with P-controller}
\label{vert_P}
\end{figure}
The rise time of the vertical PI controller is 0.4 seconds and the settling time is 7.1 seconds.
\begin{figure}[H]
\centering
\includegraphics[width = 13 cm]{Horizontal_PI_controller.png}
\caption{Evolution of horizontal angular error from \(60^{\circ}\) step with P-controller}
\label{vert_P}
\end{figure}
Rise time of horizontal PI controller is 0.6 seconds and settling time is 10.8 seconds

\subsubsection{PD controller}
\begin{figure}[H]
\centering
\includegraphics[width = 13 cm]{Vertical_PD_controller.png}
\caption{Evolution of vertical angular error from \(40^{\circ}\) step with P-controller}
\label{vert_P}
\end{figure}
The rise time of the vertical PD controller is 0.4 seconds and the settling time is 1.7 seconds.
\begin{figure}[H]
\centering
\includegraphics[width = 13 cm]{Horizontal_PD_controller.png}
\caption{Evolution of horizontal angular error from \(60^{\circ}\) step with P-controller}
\label{vert_P}
\end{figure}
Rise time of horizontal PI controller is 0.6 seconds and settling time is 0.9 seconds

\subsubsection{PID controller}
\begin{figure}[H]
\centering
\includegraphics[width = 13 cm]{Vertical_PID_controller.png}
\caption{Evolution of vertical angular error from \(40^{\circ}\) step with P-controller}
\label{vert_P}
\end{figure}
The rise time of the vertical PD controller is between 0.4 and 0.5 seconds and the settling time is 4.7 seconds.
\begin{figure}[H]
\centering
\includegraphics[width = 13 cm]{Horizontal_PID_controller.png}
\caption{Evolution of horizontal angular error from \(60^{\circ}\) step with P-controller}
\label{vert_P}
\end{figure}
Rise time of horizontal PI controller is between 0.5 and 0.6 seconds and settling time is 5.4 seconds

\subsubsection{BrickPi3 library for controlling robot arm movement}
\begin{figure}[H]
\centering
\includegraphics[width = 13 cm]{Vertical_built_in_functions.png}
\caption{Evolution of vertical angular error from \(40^{\circ}\) step with P-controller}
\label{vert_P}
\end{figure}
The rise time of the vertical PD controller is between 0.8 seconds and the settling time is 0.9 seconds.
\begin{figure}[H]
\centering
\includegraphics[width = 13 cm]{Horizontal_built_in_funtion.png}
\caption{Evolution of horizontal angular error from \(60^{\circ}\) step with P-controller}
\label{vert_P}
\end{figure}
Rise time of horizontal PI controller is between 0.8 seconds and settling time is 1 second

\subsection{Ziegler Nichols tuning results}
The controllers mentioned above was tuned with a Ziegler Nichols approach and two plots of undamped oscillations for vertical and horizontal arm movements are shown in figure \ref{vert_osc} and \ref{Hor_osc}
\begin{figure}[H]
\centering
\includegraphics[width = 13 cm]{Vertical_undamped_oscillation.png}
\caption{Evolution of vertical angular error from \(40^{\circ}\) step with P-controller}
\label{vert_osc}
\end{figure}
\begin{figure}[H]
\centering
\includegraphics[width = 13 cm]{Horizontal_undamped_oscillation.png}
\caption{Evolution of vertical angular error from \(40^{\circ}\) step with P-controller}
\label{Hor_osc}
\end{figure}

\section{Conclusions}
\label{sec:simon18}
For building a robot supposed to complete the task given in this project, the conclusion is that lego is not the best choice. Many of the lego parts are flexible and are connected to each other in a way that makes the robot very ``wobbly''. The accuracy of the robot arm was heavily reduced due to the flexibility of some lego parts. Also from section \ref{Results}, it can be seen that the fastest controller for the robot arm in terms of rise time and settling time was in fact the P controller for the vertical movement and the PD controller for horizontal movement. Despite this the included library functions from BrickPi3 for motor movement was used instead since the P and PD controllers made the robot ``wobble'' very much and there was risk of knocking the box of the industry platform. As mentioned the included library functions for motor movement was used instead since they achieved a good combination of being both quick and smooth.
For doing the task of taking a instruction from arrowhead, which was pick up the box or put down the box at different locations, our conclusion is that the robot could do it with some difficulties. The difficulties being that the lego flexibility reduced the accuracy of the robot arm which made the robot sometimes miss the box.


\appendix
\section{Team members}
The team's members are below listed (in no particular order) along with their areas of concern throughout the development of this project.
Emails are available on the title page of this document.

\begin{description}
\item[Viktor Sonesten] Repository maintainership; software packaging via
  Nix; data acquisition from the Decawave hardware; report typesetting;
  authorship of sections §\ref{intro-background}--§\ref{intro-limits},
  introduction of §\ref{sec:sys-design},
  §\ref{sec:impl}--§\ref{sec:LCM}, §\ref{sec:nodes},
  §\ref{sec:improvements}, appendix~\ref{sec:workflow}.

    \item[Lukas Karlsson]
    Embedded programming and
    Computer vision/position control.

    \item[Simon Sandberg]
    Hardware of robot, robot arm motion control;
    Forward and inverse kinematics of robot arm;
    Robot arm controller design;
    and embedded programming for robot motion; report
    typesetting; authorship of sections §1.3, §1.6, §1.6.1, §1.6.2, §1.6.3, §1.6.4, §1.7, §1.7.1, §1.7.2, §1.7.3, §1.7.4;
    sections about controllers and tuning of controllers in results section, part of robot arm/hardware/controllers in improvements;
    part about robot arm/controllers of arm in conclusion.

    \item[Oleksiy Mishchenko]
    Design and implementation of line following navigations system;
    Hardware of robot, and embedded programming for mobile platform motion; report
     typesetting; authorship of sections §1.4, §1.4.1, §1.4.2, §1.5, §1.5.1, §1.5.2, §1.7.5;
     Results: Performance Of Line Follower, Test Setup, P and PD controller for line follower, conclusion for line follower

    \item[Ruben Asplund]
    Arrowhead integration and embedded programming.
    Authorship of sections §1.4, §3.4 and the Arrowhead node in §3.3.

    \item[Ali Nouri]
    Modeling, simulation and implementation of the mobile platform;
    Control and navigation system design;
    Authorship of sections §2.1.1, §2.2.1, §3.3.

    \item[Ali Khademi]
    Dynamic equations and controller design for position.
\end{description}

\section{Work flow}
Every component of this project (code, documentation, models, simulations, etc.) is publicly hosted on GitHub at \href{https://github.com/tmplt/ed7039e}{https://github.com/tmplt/ed7039e}.

Using GitHub's auxiliary tools, each defined milestone is assigned to a GitHub repository project.
Each project contains a set of tasks that are marked as ``todo'', ``in progress'', or ``done''.
When all tasks in a project have been marked as ``done'', the milestone has been reached:
a feature has either been implemented, or a bug has been fixed, depending on the milestone definition.
For example, the milestone of implementing two-dimensional navigation is organized via the \href{https://github.com/tmplt/ed7039e/projects/1}{project under the same name}.

Once a week a meeting will be held to wrap up the previous week and make any necessary preparations for the coming week,
such as assigning tasks, helping with any issues that may have appeared, etc.

\subsection{Git work flow}
If one or more team members wishes to apply a change to the repository,
they will create an appropriate branch in their fork,
commit their changes to this branch and create a pull request to the original repository.
After its submission it is up to the repository maintainer to verify that proposed changes are:
appropriate,
hold a certain quality (e.g., in the case of a code submission, that the code is sufficiently commented and easily understood),
and may be reproducibly built. % TODO: refer to the section on Nix?
If these criteria are not met, the required changes will be communicated to the pull request author(s) either via the GitHub's built-in review tools, or in person.
When the above criteria are met, the maintainer will merge the commits into the tree of the target branch.
The main branch of this project is denoted as the ``master'' branch wherein only working code should reside.
Any features that require an extensive implementation and/or testing period may reside in any amount of additional branches,
be it under the origin repository or under a fork.

Ultimately, the maintainer is responsible for all code that is merged.


% \section{System composition}
% The current plan is to build the robot mostly with the lego ev3 parts.\\ The mobile platform will be built with the parts present in the package given to the group in the beginning of the project. The arm, which will grip the cube, will be built from external lego parts and motors, the list of these parts will be submitted furing this Tuesday. A resberry pie will be used for the communication between sensors and motors, all calculations for the control systems will also be computed int it. A preliminary model has been calculated for the forward kinematics of the system, this will probably be updated and an inverse kinematic model will be derived. The decision has not yet been made on how the location of the robot will be calculatedl, some of the possible navigation systems that have been discussed are; LIDAR, using deckawave together with a sonar, using wifi triangulation and other solutions. 

\subsection{Structure of work}
\subsubsection{Milestones}
\subsubsection{Tecchnical solutions in the project}

\printbibliography

\end{document}
